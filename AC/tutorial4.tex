\documentclass[12pt]{article}

\usepackage{amsmath}
\usepackage{amssymb}
\usepackage[a4paper, top=1in, bottom=1in, left=0.8in, right=0.8in]{geometry}

\title{Tutorial 4: Partial Derivatives}
\author{Arjun Ghosh}
\date{\today}

\begin{document}
\maketitle
\begin{enumerate}
\item 
We have the function 
$$
f(x,y) = \frac{x^{2} + y^{2}}{|x| + | y|}; (x,y) \neq (0,0)
$$
and which is zero at the origin, to clearly for all other points the function is continuous, at the origin, let us consider the expression
$$
|f(x,y)-L| = |\frac{x^{2} + y^{2}}{|x| + | y|}| = |x| + | y | -\frac{2|x||y|}{|x| + | y|} \leq | x| + | y| \leq 2\sqrt{ x^{2} + y^{2} }
$$
Where $L = 0$, now for all $\epsilon > 0$ choose $\epsilon = 2\delta$, then $\sqrt{ x^{2} + y^{2} } < \delta$ implies
$$
\epsilon = 2 \delta > 2\sqrt{ x^{2} + y^{2} } \geq |f(x,y)-L|
$$
Therefore
$$
\lim_{ (x,y) \to (0,0) } f(x,y) = 0 = f(0,0)
$$
Hence $f(x,y)$ is continuous at $(0,0)$, now consider
$$
f_{x} = \lim_{ h \to 0 } \frac{f(h,0)-f(0,0)}{h} = \frac{h^{2}}{h|h|} = \frac{h}{|h|}
$$
which clearly does not exist as its LHL and RHL are different, similarly
$$
f_{y}(0,0) = \lim_{ k \to 0 } \frac{k}{|k|}
$$
also does not exist
\item
\begin{enumerate}
\item
For this function we have 
$$
f_{x}(0,0) = \lim_{ h \to 0 } \frac{f(h,0)-f(0,0)}{h} = \frac{0-0}{h} = 0
$$
Similarly $f_y (0,0)= 0$, however, putting $y = mx^{2}$ we see that the limit at $(0,0)$ does not exist
\item
Both partial derivatives are $0$, however, along the path $y = \sin x$ we have
$$
\lim_{ (x,y) \to (0,0) } \frac{x^{3} + y^{3}}{x-y} = \lim_{ x \to 0 } \frac{x^{3} + \sin ^{3}x}{x - x + \frac{x^{3}}{3!}} = 12
$$
but along the path $x+y=0$ we have the limit $0$ hence the limit does not exist
\end{enumerate}
\item
The definition of the partial derivative is
$$
f_{x}(x_{0},y_{0}) = \lim_{ h \to 0 } \frac{f(x_{0} + h, y_{0})-f(x_{0}, y_{0})}{h}
$$
$$
f_{y}(x_{0},y_{0}) = \lim_{ h \to 0 } \frac{f(x_{0} , y_{0}+k)-f(x_{0}, y_{0})}{k}
$$

\begin{enumerate}
\item
We have
$$
f(x,y) = x^{2} + y^{2}
$$
The definition gives us
$$
f_{x}(x,y) = \lim_{ h \to 0 } \frac{(x + h)^{2} + y^{2} -x^{2} -y^{2}}{h} = \lim_{ h \to 0 } \frac{2xh + h^{2}}{h} = 2x
$$
and similarly $f_{y}(x,y)=2y$

\item
In this case we have a trigonometric with mixed arguments
$$
f(x,y) = \sin(3x + 4y)
$$
the definition gives us
$$
f_{x} = \lim_{ h \to 0 } \frac{\sin(3x + 4y + 3h)-\sin(3x + 4y)}{h} = \lim_{ h \to 0 } \frac{\cos(3x + 4y)}{h} \left( 2\sin \frac{3h}{2} \right)
$$
which is clearly
$$
f_{x} = 3\cos(3x + 4y)
$$
Similarly $f_{y} = 4\cos(3x + 4y)$

\item
We have a sum of two products
$$
f(x,y) = ye^{ -x } + xy
$$
Again, 
$$
f_{x} = \lim_{ h \to 0 }\frac{ ye^{ -x }e^{ -h }-ye^{ -x }}{h} + \frac{y(x+h-x)}{h} = -ye^{ -x }+ y
$$
Similarly
$$
f_{y} = e^{ -x } + x
$$

\item
This time we have a sum of three terms with only one function of two variables


\end{enumerate}

\item
\begin{enumerate}
\item
The function is
$$
f(x,y) = \frac{xy}{x + y}
$$
both partial derivatives at $(0,0)$ are clearly $0$ further
$$
f_{x}(0,y) = \lim_{ h \to 0 } \frac{\frac{y(h)}{y + h}-0}{h} = 1
$$
similarly
$$
f_{y}(x,0) = 1
$$
\item
Here the function is simply
$$
f(x,y) = \ln(1 +xy)
$$
now, clearly $f_{x} = f_{y} = 0$ at $(0,0)$ and $f_{x} = y$ and $f_{y}=x$ along $(0,y)$ and $(x,0)$ respectively

\item
Here the piecewise function is 1 along the axes and 0 elsewhere, clearly this means that both partial derivatives are 0 at $(0,0)$ but 
$$
f_{x}(0,y) = \lim_{ h \to 0 } \frac{f(h,y)-f(0,y)}{h} = \lim_{ h \to 0 } \frac{1}{h}
$$
if $y \neq 0$ thus the derivatives do not exist

\item
Here the function is essentially a function of one variable $u = (x-y)$
$$
e^{ u } - e^{ -u }= 2\sinh u
$$
which gives, 
$$
f_{x} = 2\cosh u_{x} = (e^{ u } + e^{ -u })u_{x}
$$
at $(0,0)$ $u=0$ 
$$
f_{x}(0,0) = 2
$$
since $u_{x}=1$ 
$$
f_{y}(0,0) = -2
$$
since $u_{y}=-1$
Further for
$$
f_{x}(0,y) = (e^{ y } + e^{ -y })(1)
$$
and 
$$
f_{y}(x,0) = (e^{ x } + e^{ -x })(-1)
$$
\item
Consider the family of curves $y = -x + mx^{3}$, along these curves, the limit at $(x,y) \to (0,0)$ becomes
$$
\lim_{ x \to 0 } \frac{x^{3}-(-x + mx^{3})^{3}}{mx^{3}} = \frac{2}{m}
$$
which clearly depends on $m$, thus the function $f(x,y)$ is not continuous at $(0,0)$, however, this does not imply that the partial derivatives do not exist, since the the function may be continuous along the axes, we have
$$
f_{x}(0,y) = \lim_{ h \to 0 } \frac{1}{h}\frac{h^{3} - y^{3}}{h + y} - \left( -\frac{y^{3}}{y} \right) = \lim_{ h \to 0 } \frac{1}{h}\left( \frac{h^{3}-y^{3}}{h + y} + y^{2}  \right)
$$
combining the fractions
$$
\lim_{ h \to 0 } \frac{h^{3}-y^{3} +hy^{2} + y^{3}}{h(h + y)} = \lim_{ h \to 0 } \frac{hy^{2}}{hy} = y
$$
similarly for $f_{y} = -x$ and both values being $0$ at $(0,0)$
\end{enumerate}

\item Show that the following functions have first order PD at $(0,0)$ and discuss their differentiability at the origin
\begin{enumerate}
\item
We have
$$
f(x,y) = \frac{xy(x^{2}-y^{2})}{x^{2} + y^{2}}
$$
Clearly this function has a really nice form in polar coordinates, 
$$
f(r, \theta) = \frac{r^{2}}{2} \cos 2\theta \sin 2\theta = \frac{r^{2}}{4} \sin 4\theta
$$
Since $\sin 4\theta \in[-1,1]$ is bounded, this function must tend to $0$ as $(r,\theta)\to (0, 0^{r})$, further it tends to zero in the second order, its derivative wrt $r$ also tends to $0$. Thus continuity is guaranteed, consider the partial derivatives at the origin, we have
$$
f_{x}(0,0) = \lim_{ h \to 0 } \frac{1}{h} \frac{hy(h^{2}-y^{2})}{h^{2} + y^{2}} -0 = 0
$$
Similarly $f_{y}(0,0) = 0$ thus to prove differentiability we need to prove that 
$$
\lim_{ (h,k) \to (0,0) } \frac{f(x + h, y + k) - f(x, y)}{\sqrt{ h^{2} + k^{2} }} = 0
$$
that is to say, since the first order partial derivatives vanish, the differential is second order or higher
$$
= \lim_{ (h,k) \to (0,0) } \frac{1}{\sqrt{ h^{2} + k^{2} }} f(h,k) = \lim_{ (h,k) \to (0,0) } \frac{hk(h^{2} -k^{2})}{(h^{2} + k^{2})^{3/2}} = 0
$$
by putting $r =  \sqrt{ h^{2} + k^{2} }$ which gives us $r \sin 4\theta$ which is bounded as before. Hence the function is indeed differentiable at the origin
\item
This function also clearly is continuous at the origin
$$
f(x,y) = \frac{x^{2}y}{x^{2} + y^{2}}
$$
and it also has a nice polar representation as
$$
f(r, \theta) = r \sin \theta \cos ^{2} \theta
$$
Which makes it tend to zero linearly but not quadratically, thus the first order partial derivatives
$$
f_{x}(0,y) = \lim_{ h \to 0 } \frac{f(h,y)-f(0,y)}{h}  = \frac{hy}{h^{2} + y^{2}} = 0
$$
while 
$$
f_{y}(x,0) = \lim_{ k \to 0 } \frac{f(x,k)-f(x,0)}{k} = \lim_{ k \to 0 } \frac{x^{2}}{x^{2} + k^{2}} = 1
$$
Thus at least one of the first order partial derivatives is non-zero, this could also have easily been seen by considering the approach angles $\theta = 0, \pi$ and $\theta = \frac{\pi}{2} , \frac{3\pi}{2}$ for $x$ and $y$ respectively. Now to check differentiability we need to prove that
$$
\lim_{ (h,k) \to (0,0) } \frac{f(h,k) - k - f(0,0)}{\sqrt{ h^{2} + k^{2} }} = 0
$$
which is
$$
\lim_{ (h,k) \to (0,0) } \frac{1}{\sqrt{ h^{2} + k^{2} }}\left( \frac{h^{2}k}{h^{2} + k^{2}}-k \right) = \frac{-k^{3}}{(h^{2} + k^{2})^{3/2}}
$$
which does not in fact exist, thus the function is non differentiable at the origin


\end{enumerate}

\item
Again we have a function linear in $r$, which will not be differentiable at the origin because of its $\theta$ dependence, 
$$
f(x,y) = \frac{xy}{\sqrt{ x^{2} + y^{2} }} = \frac{r}{2}\sin 2\theta = f(r,\theta)
$$
Now the partial derivatives at the origin are 
$$
f_{x}(0,0) = \lim_{ h \to 0 } \frac{f(h,0)-f(0,0)}{h} = \lim_{ h \to 0 } \frac{0}{\sqrt{ h^{2}  }} = 0
$$
similarly for $f_{y}(0,0) = 0$ so we need to prove that
$$
\lim_{ (h,k) \to 0 } \frac{f(x,y)-f(0,0)}{\sqrt{ h^{2} + k^{2} }} = 0
$$
but this is clearly false since the limit on the left does not exist, hence the function is not differentiable

\item
We have the function
$$
f(x,y) = \sqrt{ |xy| }
$$
We need to prove that it is continuous but non differentiable, even thought both partial derivatives exist and are zero but are discontinuous at the origin.

Away from the origin we have have $f_{x} = \frac{1}{2}\sqrt{ \frac{y}{x} }$ and $f_{y} = \frac{1}{2}\sqrt{ \frac{x}{y} }$, clearly, $f_{x}$ is zero along $y=0$ at all points other than the origin, and therefore it must also have the limit of zero along the x-axis approaching the origin, hence 
$$
f_{x}(0,0)=f_{y}(0,0) = 0
$$
Now to check differentiability we have
$$
\lim_{ (h,k) \to (0,0) } \frac{\sqrt{ |hk| }}{\sqrt{ h^{2} + k^{2} }}
$$
Clearly this limit does not exist along $k=mh$

\item
\begin{enumerate}
\item
With the function
$$
f(r, \theta) = r^{2} \sin \left( \frac{1}{r^{2}} \right)
$$
or
$$
f(x,y) = (x^{2} + y^{2})\sin \frac{1}{x^{2} + y^{2}}
$$
This will be differentiable, clearly it is continuous due to the bounding of the sine function, further, consider 
$$
f_{x}(0,0) = \lim_{ h \to 0 } \frac{f(h,0)-f(0)}{h} = \frac{h^{2} \sin\left( \frac{1}{h^{2}} \right)}{h} = 0
$$
similarly for $f_{y}(0,0)=0$, thus we have
$$
\lim_{ (h,k) \to (0,0) } \frac{h^{2} + k^{2}}{\sqrt{ h^{2} + k^{2} }}\sin \frac{1}{h^{2} + k^{2}} = 0
$$
Hence the function is differentiable at $(0,0)$
\item
$$
f(x,y) = \frac{x^{3}-2y^{3}}{x^{2} + y^{2}} = r (\cos ^{3} \theta - 2\sin ^{3} \theta)
$$

\end{enumerate}

\item First principle definition of second derivatives \\
We define
$$
f_{xx} = \lim_{ h \to 0 } \frac{f(x + 2h,y) + f(x,y) - 2f(x + h,y)}{h^{2}}
$$
iff the first derivative exists and is continuous, 
$$
f_{xy} = \lim_{ k \to 0 } \lim_{ h \to 0 } \frac{f(x+h,y+h)+ f(x,y) - f(x,y+k) -f(x + h,y)}{h^{2}}
$$
and
$$
f_{yx} = \lim_{ h \to 0 } \lim_{ k \to 0 } \frac{f(x+h,y+h)+ f(x,y) - f(x,y+k) -f(x + h,y)}{hk}
$$
note carefully the order of the limits, 
For the function
$$
f(x,y) = y \frac{x^{2} -y^{2}}{x^{2} + y^{2}}
$$
at the origin, double $x$ derivative is given by
$$
f_{xx}(0,0) = \lim_{ h \to 0 } \frac{f(2h, 0) +f(0,0) -2f(h,0)}{h^{2}} = 0
$$
because all terms are 0, for the double $y$ derivative
$$
f_{yy} = \lim_{ k \to 0 } \frac{f(0,2k) + f(0,0)-2f(0,k)}{k^{2}} = \frac{-2k + 2k}{k^{2}} = 0
$$
Now, for $f_{xy}$
$$
f_{xy}(0,0) = \lim_{ k \to 0 } \lim_{ h \to 0 } \frac{k \frac{h^{2}-k^{2} }{h^{2} + k^{2}}-k(-1)}{hk}
$$
which is 
$$
\lim_{ k \to 0 } \lim_{ h \to 0 } \frac{1}{h}\frac{h^{2} -k^{2} + h^{2} + k^{2}}{h^{2} + k^{2}} = \lim_{ k \to 0 } \lim_{ h \to 0 } \frac{2h}{h^{2} + k^{2}} = 0
$$
However, 
$$
f_{yx} = \lim_{ h \to 0 } \lim_{ k \to 0 } \frac{2h}{h^{2} + k^{2}}
$$
does not exist, clearly this means the function is not differentiable at the origin
\item We have to compare the mixed second order partial derivatives of the function
$$
\frac{x^{2}y(x-y)}{x^{2} + y^{2}}
$$
Consider the new function
$$
d(h,k) = \frac{f(h,k) + f(0,0)-f(0,k)-f(h,0)}{hk}
$$
which is
$$
d(h,k) = \frac{h(h-k)}{h^{2} + k^{2}}
$$
evidently
$$
f_{xy} = \lim_{  k \to 0 } \lim_{ h \to 0 } d(h,k) = 0
$$
while
$$
f_{yx} = \lim_{ h \to 0 } \lim_{ k \to 0 } d(h,k) = \lim_{ h \to 0 } \frac{h^{2}}{h^{2}} = 1
$$
Clearly 
$$
f_{xy}(0,0) \neq f_{yx}(0,0)
$$
and the function is in fact differentiable
\item Find the total derivative of the following functions
\begin{enumerate}
\item $w = x^{2} + xy^{2} + xy^{2}z^{3}$ \\
Clearly
$$
dw = 2xdx + y^{2}dx + 2xydy + y^{2}z^{3}dx + 2xyz^{3} dy + 3ax^{2}z^{2} dz
$$
\item $z = \arctan \frac{x}{y}$ \\
We have
$$
dz = \frac{y^{2}}{y^{2} + x^{2}}\frac{ydx - xdy}{y^{2}} = \frac{ydx -xdy}{x^{2} + y^{2}}
$$
\item $u = e^{ x^{2} + y^{2} + z^{2} }$ \\
We have
$$
\ln u = x^{2} + y^{2} + z^{2} \implies \frac{du}{u} = 2(xdx + ydy + zdz) \implies du = 2e^{ x^{2} + y^{2} + z^{2} }(xdx + ydy + zdz)
$$
\item $w=\sin(3x + 4y) + 5e^{ z }$ \\
Both terms individually give
$$
dw = \cos(3x + 4y)(3dx + 4dy) + 5e^{ z }dz
$$
\item $w = z\ln y + y\ln z + xyz$ \\
Again evaluating term by term we have
$$
dw = \ln y dz + \frac{z}{y}dy \ln z dy + \frac{y}{z} dz + yzdx + xzdy + xydz
$$
\item $u = \sqrt{ x^{2} + y^{2} + z^{2} }$ \\
Here $u^{2} = x^{2} + y^{2} + z^{2}$
$$
2udu = 2xdx + 2ydy + 2zdz \implies du = \frac{xdx + ydy + zdz}{\sqrt{ x^{2} + y^{2} + z^{2} }}
$$
\item $w = e^{ x }\sin(y + 2z)-x^{2}y^{2}$ \\

\item $w = e^{ x/y } + e^{ y/x }$  \\

\end{enumerate}
\end{enumerate}
\end{document}

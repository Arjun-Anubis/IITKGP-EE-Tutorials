\documentclass[12pt]{article}

\usepackage{amsmath}
\usepackage{amssymb}
\usepackage[a4paper, top=1in, bottom=1in, left=0.8in, right=0.8in]{geometry}

\title{Tutorial 3: Advanced Calculus}
\author{Arjun Ghosh}
\date{\today}

\begin{document}
\maketitle

\section*{Solutions to Problems}

\begin{enumerate}
	\item
	
		\begin{enumerate}
			\item 
				Clearly with $y=mx$ this limit does not exist
			\item
				Same as above
			\item
				Put $x = my^{2}$, to obtain
				$$
				L = |m|e^{ -|m| }
				$$
				which depends on m, and thus the limit does not exist
			\item
				Put $y=mx$
			\item
				Put $y=mx^{2}$, this gives us the expression
				$$
				L = \frac{mx^{4}}{x^{4 }+mx^{4}} = \frac{m}{1+m}
				$$
				which again depends on m, and thus the limit does not exist
			\item
				Put $r = mx$
			\item
				$y=mx$
			\item
				Limit does exist and is equal to $1$
			\item
				Consider $u = xy(x+y)$
				then
				$$
				f(x) = \frac{\sin u}{u} (x+y)
				$$
				Clearly this tends to $0$
			\item
				Here we have
				$$
				f(x,y) = (x-y)\left[ 1 + \frac{xy}{x^{2} + y^{2}} \right]
				$$
				which reduces to
				$$
				f(x,y) = \frac{x-y}{ \frac{x}{y} + \frac{y}{x} }
				$$
				The denominator is bounded $|D| \geq 2$ and thus this tends to $0$
			\item
				Put $z^{2} = ky = hx$
		\end{enumerate}

	\item
		\begin{enumerate}
			\item
				I will prove this proposition in more detail to outline the method of proof for the others

				We have the function
				$$
				f(x,y) = \frac{4xy^{2}}{x^{2} + y^{2}}
				$$
				and we want to prove that the limit 
				$$
				\lim_{ (x,y) \to (0,0) } f(x,y) = 0
				$$
				that is to say, the limit exists and is equal to $0$, we will use the $\epsilon-\delta$ definition of limit to prove this.
				We establish a bound on $|f(x,y)-L|$, then we use this bound to find $\delta$ such that
				$|(x,y)|<\delta_{\epsilon} \implies |f(x,y)-L| < \epsilon$

				So in this case, we consider
				$$
				|\frac{4xy^{2}}{x^{2} + y^{2}}| \leq 4\sqrt{ x^{2} + y^{2} }
				$$
				So now if $\epsilon = 4\delta$, $\sqrt{ x^{2} + y^{2} }< \delta$ implies
				$$
				\epsilon = 4\delta > 4\sqrt{ x^{2} + y^{2} } \geq |f(x,y)-L|
				$$
				which completes our proof
			\item
				Here we have to make a change of variables to use our template, 
				Let $u = x+ 1 \implies x = u-1$ and similarly for $v = y +1 \implies y = v-1$
				As before, we want to prove that
				$$
				\lim_{ (u,v) \to 0,0 } (u-1)(v-1-2u + 2)  = -1
				$$
				we can assume $|u|<1$ and $|v|<1$, now consider
				$$
				|f(u,v)-L| = |(u-1)(v-2u+1)+1| = |uv-2u^{2} + u -v + 2u|
				$$
				which is
				$$
				|uv-2u^{2} +3u-v|\leq |uv|+2|u^{2}| +3|u| + |v|
				$$
				now apply $|u|<1$ and $|u|\leq \sqrt{ u^{2} + v^{2} }$ and $|v|\leq \sqrt{ u^{2} + v^{2} }$ giving
				$$
				|f(u,v)-L| < 7\sqrt{ u^{2} + v^{2} }
				$$
				If we set $7\delta  = \epsilon$ then $\sqrt{ u^{2} + v^{2} }<\delta$ implies
				$$
				\epsilon = 7\delta > 7\sqrt{ u^{2} + v^{2} } > |f(u,v)-L|
				$$

			\item
				define $u = x-1$ so we want
				$$
				\lim_{ (u,y) \to (0,0) } f(u,y) = \frac{u^{2}\ln(1 + u)}{u^{2} + y^{2}} =0
				$$
				consider 
				$$
				|f(u,y)-L| = |\frac{u^{2}\ln(1+u)}{u^{2} + y^{2}}| = |u^{2}/(u^{2} + y^{2})| \ln(1 + u) \leq \ln(1 + u) \leq \ln(e^{ u }) = \leq \sqrt{ u^{2} + y^{2} }
				$$
				using $e^{ u}\geq 1+u$ from calculus, now if $\delta = \epsilon$, $\delta > \sqrt{ u^{2} + y^{2} }$ implies
				$$
				\epsilon = \delta > \sqrt{ u^{2} + y^{2} } \geq |f(u,y)-L|
				$$
				which completes our proof

			\item
				Let $u = x + 2$, $v = y-2$, now we need to prove that
				$$
				\lim_{ (u,v) \to (0,0) } u  -v + 4 = 4   
				$$
				now consider 
				$$
				|f(u,v)-L| = |u-v|\leq |u| + |v|\leq 2\sqrt{ u^{2} + v^{2} }
				$$
				Now if $\epsilon = 2\delta$ then $\delta > \sqrt{ u^{2} + v^{2} }$ implies
				$$
				\epsilon = 2\delta > 2\sqrt{ u^{2} + v^{2} } > |f(u,v)-L|
				$$
				Completing our proof

			\item
				Here we want to prove that
				$$
				\lim_{ (x,y) \to (0,0) } f(x,y) = xy \frac{x^{2}-y^{2}}{x^{2} + y^{2}}
				$$
				Clearly $x^{2} -y^{2}<x^{2} + y^{2}$ which gives us
				$$
				|f(x,y)-L| = |xy\frac{x^{2}-y^{2}}{x^{2} + y^{2}}| \leq |xy|
				$$
				Assume $|x|<1$ then we have
				$$
				|xy|< |y|< \sqrt{ x^{2} + y^{2} }
				$$
				now if $\delta = \epsilon$, $\delta > \sqrt{ x^{2} + y^{2} }$ implies
				$$
				\epsilon = \delta > \sqrt{ x^{2} + y^{2} } \geq |f(x,y)-L|
				$$
				which completes our proof

			\item
				We have
				$$
				f(x,y) = x \sin x \cos y
				$$
				We want to prove that
				$$
				\lim_{ (x,y) \to (0,0) } f(x,y) = L = 0 
				$$
				consider
				$$
				|f(x,y)-L| = |x\sin x\cos y|\leq |x|\leq \sqrt{ x^{2} + y^{2} }
				$$
				now if $\delta = \epsilon$, $\delta > \sqrt{ x^{2} + y^{2} }$ implies
				$$
				\epsilon = \delta > \sqrt{ x^{2} + y^{2} } \geq |f(x,y)-L|
				$$
				which completes our proof

			\item
				We have
				$$
				f(x,y) = \frac{x^{2}}{\sqrt{ x^{2} + y^{2} }}
				$$
				We want to prove that
				$$
				\lim_{ (x,y) \to (0,0) } f(x,y) = L = 0 
				$$
				consider
				$$
				|f(x,y)-L| =\frac{x^{2}}{\sqrt{ x^{2} + y^{2} }}|\leq \sqrt{ x^{2} + y^{2} }
				$$
				since, $x^{2} \leq x^{2} + y^{2}$, now if $\delta = \epsilon$, $\delta > \sqrt{ x^{2} + y^{2} }$ implies
				$$
				\epsilon = \delta > \sqrt{ x^{2} + y^{2} } \geq |f(x,y)-L|
				$$
				which completes our proof
			\item
				We have
				$$
				f(x,y) = \frac{x^{2}y^{2}}{x^{2} + y^{2}} 
				$$
				We want to prove that
				$$
				\lim_{ (x,y) \to (0,0) } f(x,y) = L = 0 
				$$
				consider
				$$
				|f(x,y)-L| = \frac{x^{2}y^{2}}{x^{2} + y^{2}} < x^{2} + y^{2} < \sqrt{ x^{2} + y^{2} }
				$$
				if we assume $x^{2} + y^{2} < 1$ now if $\delta = \epsilon$, $\delta > \sqrt{ x^{2} + y^{2} }$ implies
				$$
				\epsilon = \delta > \sqrt{ x^{2} + y^{2} } \geq |f(x,y)-L| 
				$$
				which completes our proof
		\end{enumerate}
\end{enumerate}


\end{document}

\documentclass[12pt]{article}

\usepackage{amsmath}
\usepackage{amssymb}

\title{Problem Set 6}
\author{Arjun Ghosh}
\date{\today}

\begin{document}
\maketitle
\section{Question 1}
Consider two $x$ values, $x=0$ and $x = L$ where $L \to \infty$, clearly
$$
f(x) = 2x^{3} + 5x -9  = -9
$$ at $x=0$ and 
$$
f(L) \to \infty \text{ as } L \to \infty
$$
thus $\exists \,x \in [0,\infty) :f(x) = 0$ hence, $f(x)$ has at least one root, further, for the sake of contradiction let us assume there exist two or more roots of $f(x)$, let any two of them be $x_{1}$ and $x_{2}$, then by Rolle's theorem
$$
f'(x) = 6x^{2} + 5 = 0
$$
for some $c \in (x_{1}, x_{2})$ which is clearly absurd, thus our assumption must be false, and thus there is only one root of $f(x)$

\section{Question 2}
\subsection{Part (a)}
Consider $f(x) = |x-1|$ in the interval $[0,2]$, clearly $f(x)$ is not differentiable at $x=1$, since $f'(1^{-}) = -1$ and $f'(1^{+})=1$, thus it does not satisfy the conditions for Rolle's theorem and it doesn't satisfy its consequences either since $\not\exists \, x \in (0,2) : f'(x) = 0$

\subsection{Part (b)}
$$
f(x) = 1 + x^{2/3}
$$
in the interval $[-8,8]$, simply differentiating we have
$$
f'(x) = \frac{2}{3}x^{-1/3}
$$
which is not defined for $x=0$ and clearly the derivative is not continuous at $x=0$ anyways, thus it does not satisfy the conditions for Rolle's theorem and clearly there does not exist $c \in[-8,8]: f'(c) = 0$

\subsection{Part (c)}
$$
f(x) = x \sin \frac{1}{x}; f(0) = 0
$$
Consider
$$
f'(0) = \lim_{ h \to 0 } \frac{\left( h\sin\left( \frac{1}{h} \right)  \right)}{h} = \lim_{  h \to 0 } \sin\left( \frac{1}{h} \right)
$$
does not exist, hence it does not satisfy the conditions for Rolle's theorem, differentiating on the differentiable intervals
$$
f'(x) = \sin\left( \frac{1}{x} \right) - \frac{x\cos\left( \frac{1}{x} \right)}{x^{2}} = \sin\left( \frac{1}{x} \right) -\frac{1}{x}\cos\left( \frac{1}{x} \right)
$$
Setting it to $0$
$$
\tan\left( \frac{1}{u} \right) = \frac{1}{u}
$$
There are infinitely many solutions to this equation in the interval $\frac{1}{u} \in\left( -\frac{\pi}{2}, \frac{\pi}{2} \right)$ or $u \in \left( \infty, -\frac{2}{\pi} \right) \cup\left( \frac{2}{\pi}, u \right)$, so there do exist value of $c$ in the interval which satisfy the consequences of Rolle's theorem even though its conditions are not satisfied.

\section{Question 3}
\subsection{Part (a)}
In the interval $\left[ \frac{\pi}{4}, \frac{3\pi}{4} \right]$, Cauchy's MVT gives us
$$
\frac{f'(\xi)}{g'(\xi)} = \frac{1-1}{1-(-1)} = 0 = -\frac{\cos(\xi)}{\sin(\xi)}
$$
Clearly $\xi = \frac{\pi}{2}$ satisfies this condition

\subsection{Part (b)}
In the interval $\left[ 0,  \frac{1}{2} \right]$ Cauchy's MVT gives us
$$
\frac{f'(\xi)}{g'(\xi)} = \frac{\frac{3}{2}(1+\xi)^{1/2}}{\frac{1}{2}(1+\xi)^{-1/2}} = 3(1+\xi) = \frac{f(b)-f(a)}{g(b) -g(a)} 
$$
which gives us
$$
3(1+\xi) = \frac{\left( u \right)^{3/2}-1}{\sqrt{ u }-1} \implies \xi = \frac{1}{3} (u + \sqrt{ u } + 1)-1
$$
where $u = \frac{3}{2}$

\section{Question 4}
We write LMVT as: for an interval $[a,b]$ if $f(x)$ is continuous and differentiable in the corresponding open interval $(a,b)$, then $\exists \, c \in (a,b) : f'(c) = \frac{f(b)-f(a)}{b-a}$

Now define $\theta = \frac{c-a}{b-a}$, since 
$$
c \in(a,b) \implies c-a \in(0,b-a) \implies \theta \in(0,1)
$$
Thus we have
$$
\boxed{
f'(c) = f'(a + c-a) = f'(a + \theta(b-a)) = \frac{f(b)-f(a)}{b-a}
}
$$
Now we set $b-a = h$ and $a = x$ so we have
$$
f'(x + \theta h) = \frac{f(x + h)-f(x)}{h}
$$

Now, before we continue, let us examine this equation, if we apply Taylor's theorem to $f(x +h)$, $i$.$e$. in the interval $(x,x+h)$ we find $\theta_{2} \in (0,1)$ such that
$$
hf'(x + \theta h) = \left( f(x) + hf'(x) + \frac{h^{2}}{2}f''(x + \theta_{2}h) \right)-f(x)
$$
or
$$
\frac{f'(x + \theta h)-f'(x)}{h} =\frac{1}{2}f''(x +\theta_{2}h)
$$
Now if we take the limit as $h \to 0$
$$
\lim_{ h \to 0 } \frac{f'(x + \theta h)-f(x)}{h} = \theta f''(x) = \lim_{  h \to 0 } \frac{1}{2}f''(x + \theta_{2}h)
$$
which implies
$$
\lim_{ h \to 0 } \theta = \frac{1}{2}
$$
as a consequence of Taylor's theorem, this however requires the existence of the second derivative of $f(x)$ at $x$.

Clearly, $\theta = g(x,h)$, $i$.$e$. $\theta$ is a function of $x$ and $h$, thus

\subsection{Part (a)}
for $f(x) = x^2$
We have
$$
f(x + h) = (x+h)^{2} = x^{2} + 2xh + h^{2}
$$
or 
$$
\frac{f(x + h)-f(x)}{h} = 2x + h = f'(x + \theta h) = 2(x + \theta h) = 2x + 2\theta h
$$
Clearly
$$
\theta = \frac{1}{2} \, \forall \,x, h
$$
which matches what we had earlier
\subsection{Part (b)}
$$
\frac{f(x + h)-f(x)}{h} = e^{ x }\left(  \frac{e^{ h }-1}{h} \right)
$$
While
$$
f'(x + \theta h) = e^{ x }e^{ \theta h }
$$
Thus we have
$$
\theta = \frac{1}{h} \ln \frac{e^{ h }-1}{h}
$$
In this case we have $\theta$ not dependent on $x$, a graph of this function is given, taking the limit as
$$
\theta_{0} = \lim_{ h \to 0 } \frac{1}{h} \ln \frac{e^{ h }-1}{h} = \lim_{ h \to 0 } \frac{1}{h}\ln \frac{h + \frac{h^{2}}{2} +O(h^{3})}{h}
$$
which is
$$
\theta_{0} = \lim_{ h \to 0 }\frac{1}{h} \ln \left( 1 + \frac{h}{2} + O(h^{2}) \right) = \lim_{ h \to 0 } \frac{1}{2} + O(h)  = \frac{1}{2}
$$
which matches what we had earlier.

\subsection{Part (c)}
$$
\frac{f(x+h)-f(x)}{h} = \frac{1}{h}\ln \left( 1 + \frac{h}{x}\right) = f'(x + \theta h) = \frac{1}{x + \theta h}
$$
or we have
$$
\theta = \frac{1}{h}\left( \frac{h}{\ln \left( 1 + \frac{x}{h} \right)}-x \right)
$$
This time, $\theta$ is a function of $x$ and $h$, our claim is that the dependence on $x$ dies out as $h \to 0$ we have
$$
\theta = \lim_{ h \to 0 } \frac{1}{h}\left( \frac{h}{\left( \frac{x}{h}  \right)-\frac{1}{2}\left( \frac{x}{h} \right)^{2} +O\left( \frac{h^{3}}{x^{3}} \right)}  -x\right)
$$
which is
$$
\lim_{ h \to 0 } \frac{h}{x}\left( \frac{1}{1-\frac{1}{2}\left( \frac{x}{h} \right)+O\left( \left( \frac{x}{h} \right)^{2} \right)}-1 \right) = \frac{1}{2}
$$
which confirms our claim

\section{Question 5}
\subsection{Part (a)}
By LMVT, since $f$ is continuous and differentiable, for $c \in [1,2]$
$$
f(2) = f(1) + f'(c)(1)
$$
so
$$
f(1) = f(2) - f'(c) = -5 -f'(c)
$$
but this lies between
$$
f(1) \in [-7,-3]
$$
since $f'(c) \in [-2,2]$ and this maximum value is achieved if $f(x)$ is linear in the interval $(1,2)$

\subsection{Part (b)}
We know that
$$
28^{1/3} = (27 +1)^{1/3} = 3\left( 1 + \frac{1}{27} \right)^{1/3}
$$
applying LMVT we have
$$
\left( 1 + u \right)^{1/3} = 1 + \frac{u}{3} (1+ v)^{-2/3}
$$
where $u = \frac{1}{27}$ and $v \in (0,u)$, let us assume without proof for now that $(1+v)^{-2/3} = 1 + O(u)$, which is a consequence of Taylor's theorem. So we have
$$
(28)^{1/3} = 3 + \frac{1}{27} + O\left( \left( \frac{1}{27} \right)^{2} \right) \approx \frac{82}{27}
$$

\subsection{Part (c)}
We have $f''(x) \geq 0$ for $x \in [a,b]$, Let us assume for the sake of contradiction that
$$
f\left( \frac{x_{1} + x_{2}}{2} \right) > \frac{1}{2}[f(x_{1}) + f(x_{2})]
$$
for *some* $x_{1},x_{2} \in[a,b]$. Assume WLOG $x_{2}> x_{1}$, now consider the point $\bar{x} = \frac{1}{2}(x_{1} + x_{2})$, also define $\Delta x  = \bar{x} -x_{1} = x_{2} -\bar{x}$, thus $2\Delta x = x_{2} -x_{1} > 0$ hence $\Delta x > 0$

then by LMVT on the intervals $(x_{1},\bar{x})$ and $(\bar{x}, x_{2})$ (whose conditions are satisfied by the assumption that $f''(x)$ exists and is positive) yield $\xi_{1} \in(x_{1},\bar{x})$ and $\xi_{2} \in(\bar{x}, x_{2})$ and thus $\xi_{2} > \xi_{1}$ such that
$$
f'(\xi_{1}) = \frac{f(\bar{x})-f(x_{1})}{x-x_{1}}; f'(\xi_{2}) = \frac{f(x_{2})-f(\bar{x})}{x_{2} - \bar{x}}
$$
which we rewrite as
$$
f(\bar{x}) = f(x_{1}) + f'(\xi_{1})\Delta x = f(x_{2}) -f'(\xi_{2})\Delta x
$$
Adding both RHS sides and doubling the LHS
$$
2f(\bar{x}) = f(x_{1}) + f(x_{2}) + \Delta x [f'(\xi_{1})-f(\xi_{2})]
$$
Or
$$
f(\bar{x}) = \frac{1}{2}(f(x_{1}) + f(x_{2}))  + \frac{\Delta x}{2} (f'(\xi_{1})-f'(\xi_{2}))
$$
But our assumption now implies
$$
\frac{\Delta x}{2} (f'(\xi_{1})-f'(\xi_{2})) >0
$$
but since $\Delta x >0$ and $\xi_{2} > \xi_{1}$ this contradicts the statement that $f''(x) > 0$ for $[a,b]$, thus our assumption must be false and thus
$$
\boxed{
f\left( \frac{x_{1} + x_{2}}{2} \right) \leq \frac{1}{2}[f(x_{1}) + f(x_{2})]
}
$$
Completing our proof

\section{Question 6}
\subsection{Part (a)}

To prove the right hand side of the inequality we apply CMVT to $\sin x$ and $x$ on the interval $[0,x]$ to get
$$
\cos \xi = \frac{\sin x}{x} \implies \frac{\sin x}{x}<1 \implies \sin x < x;\, \forall\, x \in\left( 0, \frac{\pi}{2} \right)
$$
For the LHS, consider the function
$$
\zeta(x) = \frac{x}{\sin x} \implies \zeta'(x) = \frac{\sin x -x\cos x}{(\sin x)^{2}} > 0
$$
since $\tan x > x$ in $x \in\left( 0, \frac{\pi}{2} \right)$, therefore
$$
\zeta(x) < \zeta\left( \frac{\pi}{2} \right) = \frac{\pi}{2}; \forall x \in \left( 0, \frac{\pi}{2} \right)
$$
which gives
$$
\boxed{
\frac{2x}{\pi} < \sin x < x
}
$$
(b)
Consider the functions $x^n$ and $x$ on the interval $[a,b]$, clearly CMVT gives
$$
n\xi^{n-1} = \frac{b^{n}-a^{n}}{b-a}
$$Now if $n>1$ then $n-1>0$ which implies that $x^{n-1}$ is an increasing function which implies
$$
na^{n-1}< n\xi^{n-1} < nb^{n-1}
$$
which gives us the inequality
$$
\boxed{
na^{n-1} < \frac{b^{n} - a^{n}}{b-a} < nb^{n-1}
}
$$
(c)
For the right hand side consider the functions $\log(1+x)$ and $x$ on the interval $[0,x]$ CMVT gives
$$
1>\frac{1}{1+ \xi} =\frac{\log(1+x)}{x}
$$
which gives 
$$
\log(1+x)< x
$$
apply LMVT to the function $\log(1+x)$ on the interval $[0,x]$ gives
$$
\log(1+x) = 0 + \frac{x}{1 + \xi} > \frac{x}{1+x}
$$since $\xi < x$ or
$$
\boxed{
\frac{x}{1+x}<\log(1+x)<x
}
$$
$\forall x>0$

7(a)
Let the intersection happen at some point $\eta \in (a,b)$, now apply LMVT on $(a,\eta)$ and $(\eta, b)$ to give $\xi_{1}, \xi_{2}:\xi_{2}>\xi_{1}$ such that
$$
f'(\xi_{1})=f'(\xi_{2}) = m = \frac{f(b)-f(a)}{b-a}
$$
now apply Rolle's theorem on $f'(x)$ on the interval $(\xi_{1},\xi_{2})$
$$
f''(\zeta) = 0
$$
for some $\zeta \in (\xi_{1}, \xi_{2}) \subset (a,b)$

(b) Apply CMVT to $f(x)$ and $x^{2}$ on the interval $[0,1]$ to give
$$
\frac{f'(\xi)}{2\xi} = f(1)-f(0)
$$
for $\xi \in(0,1)$

(c) Consider $g(x)= f(x)-x$ then $g'(x) = f'(x)-1$, so we have
$$
g(a) = g(b) = 0 \implies g'(\xi) = 0
$$
for some $\xi \in(a,b)$ , Now apply LMVT to $(a,\xi)$ and $(\xi, b)$ to find $\zeta_{1}, \zeta_{2}$ such that
$$
g'(\zeta_{1}) = \frac{g(\xi)}{\xi-a}, g'(\zeta_{2}) = \frac{-g(\xi)}{b-\xi}
$$
which must have opposite signs, thus applying IVT between $(\zeta_{1}, \xi)$ and $(\xi,\zeta_{2})$ we can find two numbers $c_{1}, c_{2}$ such that
$$
\boxed{
g'(c_{1})+ g'(c_{2}) = 0 \implies f'(c_{1}) + f'(c_{2}) = 2
}
$$
8 (a)
We want to prove that there exists $c \in(a,b)$
$$
\begin{vmatrix}
f(a) + (b-a)f'(c) & f(b) \\
\phi(a) + (b-a)\phi'(c) & \phi(b) 
\end{vmatrix}=0
$$
Consider CMVT on $f(x)$ and $\phi(x)$ on the interval $(a, b)$ we have
$$
f(b) = f(a) + \frac{f'(c)}{\phi'(c)} (\phi(b)-\phi(a))
$$
(b) Apply CMVT on the functions $f(x)$, $x$, $x^{2}$, $x^{3}$ to get
$$
f(b)-f(a) = f'(x_{1})(b-a) = \frac{f'(x)}{2x_{2}}(b^{2}-a^{2}) = \frac{f'(x_{3})}{3x_{3}^{2}}(b^{3}-a^{3})
$$
or
$$
f'(x_{1})=\frac{f'(x_{2})}{2x_{1}} = \frac{f'(x_{3})}{3x_{3}^{2}}
$$
9
(a)
Consider $1-\cos x = 2\sin ^{2} \frac{x}{2}$, setting $u = x/2$ we need to prove 
$$
\sin x < x
$$
which we can do by applying CMVT to these functions
$$
\cos \xi = \frac{\sin x }{x}
$$
which completes our proof

(b) consider $\frac{f(x)}{x^{2}}$ and $\frac{1}{x^{2}}$ applying CMVT we have
$$
\frac{\frac{f(b)}{b^{2}}-\frac{f(a)}{a^{2}}}{\frac{1}{b^{2}}-\frac{1}{a^{2}}} = \frac{\frac{c^{2}f'(c) - 2cf(c)}{c^{4}}}{-\frac{2}{c^{3}}}
$$
$$
\frac{a^{2}f(b)-b^{2}f(a)}{a^{2}-b^{2}} = \frac{1}{2} (2f(c)-cf'(c))
$$
as required
(c)
Apply CMVT on the functions $\ln x$ and $\arcsin x$ on the interval $[x,1]$ to get
$$
\frac{-\ln x}{\frac{\pi}{2}-\arcsin x} = \frac{2\ln x}{2\arcsin x-\pi} = \sqrt{ 1-\frac{1}{\xi^{2}} }
$$
which is an increasing function for $\xi$ so we have
$$
\frac{2\ln x}{2\arcsin x-\pi} < \frac{\sqrt{ 1-x^{2} }}{x}
$$
for $x \in (0,1)$

10
Apply LMVT between $(a,x_{0})$ and $(x_{0}, b)$ to find $\xi_{1}<\xi_{2}$ such that
$$
f'(\xi_{1}) = \frac{f(x_{0})}{x_{0}-a}; f(\xi_{2}) = \frac{-f(x_{0})}{b-x_{0}}
$$
Now assume for the sake of contradiction that there does not exist $c \in (a,b)$ such that $f''(c)<0$ which would imply $f'(x)$ is an increasing function which implies
$$
f'(\xi_{2}) > f'(\xi_{1}) \text{ since } \xi_{2}>\xi_{1}
$$
but this contradicts what we found and thus our assumption, and therefore our assumption must be wrong and there exists $c \in (a,b)$ such that $f''(c)<0$

\end{document}
